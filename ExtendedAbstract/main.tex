\documentclass[letterpaper, 10 pt, conference]{ieeeconf}  % Comment this line out if you need a4paper

%\documentclass[a4paper, 10pt, conference]{ieeeconf}      % Use this line for a4 paper

\IEEEoverridecommandlockouts                              % This command is only needed if 
                                                          % you want to use the \thanks command

\overrideIEEEmargins                                      % Needed to meet printer requirements.

% See the \addtolength command later in the file to balance the column lengths
% on the last page of the document

\usepackage{times}
\usepackage{cite}
\usepackage{graphics} % for pdf, bitmapped graphics files
\usepackage{graphicx}
\usepackage{epsfig} % for postscript graphics files
\usepackage{mathptmx} % assumes new font selection scheme installed
%\usepackage{times} % assumes new font selection scheme installed
\usepackage{amsmath} % assumes amsmath package installed
\usepackage{amsfonts}
\usepackage{amssymb}  % assumes amsmath package installed
\usepackage{bm}
% \usepackage{todonotes}
\usepackage{url}
\usepackage{multicol}
\usepackage[bookmarks=true]{hyperref}
\usepackage{verbatim}
\usepackage{color}
\usepackage{tabularx}
\usepackage{booktabs}
\usepackage{listings}
\usepackage[utf8]{inputenc}
\usepackage{pifont}
\usepackage{mathtools}
\usepackage{breqn}
\usepackage{xcolor}
\usepackage{setspace}

\usepackage{atbegshi} % to hook to shipout

\newcommand{\cmark}{\ding{51}}%
\newcommand{\xmark}{\ding{55}}%

%\makeatletter
%\def\endthebibliography{%
%  \def\@noitemerr{\@latex@warning{Empty `thebibliography' environment}}%
%  \endlist
%}
%\makeatother

\input{00-settings}

% ### uncomment to produce preprint IEEE copyright notice ###
%\AtBeginShipout{%
%  \AtBeginShipoutUpperLeft{%
%    \put(\dimexpr.5\paperwidth-9cm\relax,-1cm){%
%      \parbox{\textwidth}{\centering\footnotesize%
%        © 2023 IEEE. Personal use of this material is permitted. Permission from IEEE must be obtained for all other uses, in any current or future media, including reprinting/republishing this material for advertising or promotional purposes, creating new collective works, for resale or redistribution to servers or lists, or reuse of any copyrighted component of this work in other works.}%
%    }%
%  }%
%}

\begin{document}
	
\title{\LARGE \bf
LRHControl: A Software Ecosystem for Research in Reinforcement Leaning-based Receding Horizon Control\authorrefmark{3}
}

\author{
  \authorblockN{%
    Andrea Patrizi\authorrefmark{1}\authorrefmark{2}, Carlo Rizzardo\authorrefmark{1}, 
    and Nikos G. Tsagarakis\authorrefmark{1}
  }
}

\maketitle

\begingroup\renewcommand\thefootnote{\authorrefmark{2}}
	\footnotetext{Department of Informatics, Bioengineering, Robotics and Systems Engineering, Università di Genova, Via All'Opera Pia 13, 16145 Genova.}
\endgroup

\begingroup\renewcommand\thefootnote{\authorrefmark{1}}
	\footnotetext{Humanoids and Human-Centred Mechatronics (HHCM), Istituto Italiano di Tecnologia (IIT), Via San Quirico 19d, 16163 Genova.}
\endgroup

\begingroup\renewcommand\thefootnote{\authorrefmark{3}}
\footnotetext{This project has received funding from the European
	Union’s Horizon Europe Framework Programme under grant
	agreement No 101070596
}
\endgroup

\setlength{\textfloatsep}{12.0pt plus 8.0pt minus .0pt}

\begin{abstract}
Robotics research in locomotion is undergoing a transformative shift towards the use of learning-based tools. Learning methodologies have been shown to be capable of remarkable robustness and performance even when applied to real-world environments; however, they present limitations in interpretability, safety guarantees and sample efficiency. For this reason, it is the authors' belief that more classical control approaches should not be disregarded yet. We thus advocate for a hybrid approach, combining offline data-based policy design through Reinforcement Learning (RL), with classical online Motion Planning, via Receding Horizon Control (RHC). Even though this kind of hybrid approaches are not entirely new, to the authors' knowledge, there is no specific tool currently available for research in this domain. To this purpose, we developed a modular software ecosystem, hereby briefly presented in its main components and features. 
%We care to stress that the framework is currently under active development, and features might not be stable or could be lacking. 
To facilitate its usability and diffusion, we made all the core components open source under the GPLv2 license. Furthermore, to showcase the potential of our framework and approach, we briefly present a proof-of-concept example combining a high-level RL agent coupled with a lower-level MPC controller for the execution of a simple locomotion task on a simulated quadruped robot.
\end{abstract}

\IEEEpeerreviewmaketitle

\section{A Hystorical Overview: \textnormal{\textit{from Markov Decision Processes and Dynamic Programming to modern Receding Horizon Control and Reinforcement Learning}}}
The foundation of \textbf{Markov Decision Processes} (MDPs) can be traced back to the work of Andrey Markov, a Russian mathematician, who developed the theory of Markov chains in the early 20th century
MDPs. In the late 1950s and early 1960s, Richard Bellman, along with his colleague Howard Dreyfus, introduced MDPs as a mathematical framework for sequential decision-making under uncertainty~\cite{rl:bellman1957markovian}.
MDPs consist of a set of state $S$, representing the possible configurations or conditions of the system being modeled. At any given time, the system is in one of these states. For each state in the MDP, there is a set of possible actions $A$ that the decision-maker can choose from. Upon taking an action $a\in\,A$ in a particular state $s\in\,S$, the system transitions to a new state $s'\in\,S$ according to a probability distribution $P(s' \mid s, a)$. These transition probabilities capture the stochastic nature of the environment and its dynamics. Associated with each $\{s,\,a\}$ pair is an immediate reward $r(s,\,a)$, expression of the immediate benefit or cost incurred by the decision-maker upon taking that action in that state. The objective in MDPs is to find an optimal policy $\pi(a \vert s)$ that maximizes the cumulative reward obtained by the decision-maker over time. This involves balancing the trade-off between immediate rewards and long-term benefits, taking into account the probabilistic nature of the environment and the dynamics of the system.
In the late 1950s, Richard Bellman introduced the so-called \textbf{Bellman equation}:\\
\textit{Let $\pi^*$ be an optimal policy for a Markov Decision Process (MDP) with state space $S$, action space $A$, state transition probabilities $P(s' \mid s, a)$, and immediate rewards $r(s, a)$. Then, for any state $s$ in the state space $S$, the following equation holds:
	\[
	V^*(s) = \max_{a \in A} \left[ r(s, a) + \gamma \sum_{s' \in S} P(s' \mid s, a) V^*(s') \right]
	\]
	where:
	\begin{itemize}
		\item $V^*(s)$ is the optimal value function for state $s$,
		\item $r(s, a)$ is the immediate reward received upon taking action $a$ in state $s$,
		\item $P(s' \mid s, a)$ is the probability of transitioning to state $s'$ after taking action $a$ in state $s$,
		\item $\gamma$ is the discount factor representing the importance of future rewards, and
		\item $\max_{a \in A} [\cdot]$ denotes the maximum over all possible actions in state $s$.
\end{itemize}}
Building upon the formulation of MDPs and the above Bellman Equation, Richard Bellman and other researchers developed the theory of Dynamic Programming (DP) as a systematic method for solving optimization problems by breaking them down into simpler subproblems~\cite{rl:bellman1960dynamic}.

\cite{psyc:skinner2019behavior}

\cite{rl:rumelhart1986learning}
\cite{rl:kakade2001natural}
\cite{rl:peters2005natural}
\cite{rl:degris2012off}
\cite{rl:schulman2015trust}
\cite{rl:schulman2017proximal}
\cite{rl:pardo2018time}
\cite{rl:haarnoja2018soft}
\cite{rl:makoviychuk2021isaac}
\cite{rl:rudin2022learning}
\cite{rl:schneider2023learning}
\cite{rl:mujocoaccelereted2023}
\cite{rl:miki2024learning}

\cite{frameworks::horizon_to}

\cite{web::lrhc_boston_dyn}
\cite{frameworks:mittal2023orbit}
\cite{frameworks:howell2022}
\cite{mpc_learn:hewing2020learning}
%\section{RL versus RHC: \textnormal{\textit{formulation}}}
\begin{itemize}
	
	\item Immediate rewards: $R(s,\,a,\,s')$
	
	A policy function mapping states to probability distributions over actions $\pi(a \vert s)$.
	
\end{itemize}
%\section{RL versus RHC: \textnormal{\textit{shortcomings and advantages}}}
\section{A hybrid approach: \textnormal{\textit{Learning-Based Receding Horizon Control with Reinforcement Learning}}}
Most of the currently employed control tools rely either on online ``model-based'' controllers~\cite{modern_mpc:neunert2018whole,web::atlas_grip_boston_dyn} (e.g. Receding Horizon Control) or ``model-free'' learned policies (e.g. Reinforcement Learning)~\cite{mpc_learn:aswani2012provably, mpc_learn:terzi2018learning, mpc_learn:soloperto2018learning, rl:schneider2023learning, rl:miki2024learning,mpc_learn:berkenkamp2016safe,mpc_learn:marco2016automatic,mpc_learn:brunner2015stabilizing,mpc_learn:rosolia2019learning,mpc_learn:englert2017inverse,mpc_learn:koller2018learning,mpc_learn:wabersich2021probabilistic,mpc_learn:gillulay2011guaranteed,mpc_learn:wabersich2018safe,mpc_learn:berkenkamp2017safe}, which are usually trained offline. Both approaches are historically deeply rooted in DP and MDPs. 
In the past years there have been several attempts at combining-learning based method and receding horizon controllers~\cite{mpc_learn:tsounis2020deepgait,mpc_learn:gangapurwala2021real}. Specifically, the following main approaches can be identified~\cite{mpc_learn:hewing2020learning}:
\begin{itemize}
	\item[1)] \textit{Model augmentation}: integration of learned models into RHC controllers to improve prediction accuracy and control performance~\cite{mpc_learn:aswani2012provably,mpc_learn:terzi2018learning,mpc_learn:soloperto2018learning}.
	\item[2)] \textit{Adaptive tuning} and \textit{parameter optimization}: RHC parameters tuning (e.g. weights, costs, constraints), based on real-time data~\cite{mpc_learn:berkenkamp2016safe,mpc_learn:marco2016automatic,mpc_learn:brunner2015stabilizing,mpc_learn:rosolia2019learning,mpc_learn:englert2017inverse}.
	\item[3)] \textit{Safety}: a learned-policy is coupled with a RHC controller, which in this context takes the role of a \textit{safety filter}~\cite{mpc_learn:koller2018learning,mpc_learn:wabersich2021probabilistic,mpc_learn:gillulay2011guaranteed,mpc_learn:wabersich2018safe,mpc_learn:berkenkamp2017safe}.
\end{itemize}
\section{Implementation: \textnormal{\textit{available framework overview and main components}}}
\begin{figure}[t]
	\centering
	\includegraphics[width=0.9\columnwidth]{imgs/cocluster_arch.pdf}
	\caption{High-level overview of the software implementation of the training environment to which the agent is exposed: the robot in the simulator is controlled through a joint-level impedance controller, which is in turn used by a higher-level receding horizon controller. The agent can indirectly control the robot through the latter. The training environment lives in an independent process and uses shared memory for interacting with the simulation environment.}
	\label{fig:coclbridge_arch}
\end{figure}
The advent of accurate GPU-accelerated simulation tools~\cite{web::isaacsim,rl:mujocoaccelereted2023} allows for a massive decrease in the training wall-time for data-hungry algorithms like RL and facilitates real-world deployment through domain randomization~\cite{rl:rudin2022learning,rl:rudin2022advanced}. We consequently choose \textit{Omniverse IsaacSim}~\cite{web::isaacsim} as the simulation backend, while we use PyTorch for all deep-learning related components and  \textit{Horizon}~\cite{frameworks::horizon_to} for formulating and running RHC controllers (on CPU, with an iLQR solver backend). One big drawback of this approach is the presence of the controllers on CPU, which currently represents our bottleneck in terms of training performance and parallelization capacity.
Fig.~\ref{fig:coclbridge_arch} shows a high level software overview of the main moduli the echosystem is made of. Specifically, we developed the following software modules:
\begin{itemize}
	\item \textit{SharsorIPCpp}~\cite{mystuff::sharsoripcpp} serves as the shared memory backend for fast data sharing and synchronization between all components on CPU.
	\item \textit{OmniRoboGym}~\cite{mystuff::omnirobogym} is used as a wrapper around IsaacSim and provides an interface to the \textit{simulation} environment.
	\item \textit{CoClusterBridge}~\cite{mystuff::coclusterbridge} exploits~\cite{mystuff::sharsoripcpp} and coordinates the connection and synchronization between the simulation environment and a cluster of RHC controllers. It furthermore provides abstractions for the controllers and an extensible debug GUI for monitoring the cluster.
	\item \textit{LRHControl}~\cite{mystuff::lrhccontrol} is the main package of the echosystem and is responsible for setting up and running the simulation environment, the control cluster and the training environment.
	\item \textit{RHCViz}~\cite{mystuff::rhcviz} is a debug tool based on ROS1/ROS2 and RViz for visualizing RHC solutions in real-time. For our specific use case, it also allows to inspect a single environment during training without the need of any rendering on the simulator side.
\end{itemize}





\section{A proof-of-concept example: \textnormal{\textit{learning acyclic stepping for locomotion}}}
\begin{figure}[t]
	\centering
	\includegraphics[width=0.9\columnwidth]{imgs/proof_of_concept.pdf}
	\caption{Preliminary results showing the agent learning to move the robot forward exploiting the underlying RHC controller, from right to left and from top to bottom.}
	\label{fig:proof}
\end{figure}
To showcase the potential of the proposed hybrid RL-RHC approach and of our framework, we trained a RL agent using PPO~\cite{rl:schulman2017proximal} to exploit a RHC controller for achieving a very simple forward locomotion task on a quadruped robot (shown in Fig.~\ref{fig:proof}). The agent is given a forward velocity reference to track and is continuously rewarded based on the task error and the performance of the underlying RHC controller. Notably, we observe the emergence of completely acyclic contact phases, varying from crawling to bound-like patterns.
 
\input{07-future_work}
\bibliographystyle{ieeetr}
\bibliography{bibliography/refs}

\end{document}
