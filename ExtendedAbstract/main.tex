\documentclass[letterpaper, 10 pt, conference]{ieeeconf}  % Comment this line out if you need a4paper

%\documentclass[a4paper, 10pt, conference]{ieeeconf}      % Use this line for a4 paper

\IEEEoverridecommandlockouts                              % This command is only needed if 
                                                          % you want to use the \thanks command

\overrideIEEEmargins                                      % Needed to meet printer requirements.

% See the \addtolength command later in the file to balance the column lengths
% on the last page of the document

\usepackage{times}
\usepackage{cite}
\usepackage{graphics} % for pdf, bitmapped graphics files
\usepackage{graphicx}
\usepackage{epsfig} % for postscript graphics files
\usepackage{mathptmx} % assumes new font selection scheme installed
%\usepackage{times} % assumes new font selection scheme installed
\usepackage{amsmath} % assumes amsmath package installed
\usepackage{amsfonts}
\usepackage{amssymb}  % assumes amsmath package installed
\usepackage{bm}
% \usepackage{todonotes}
\usepackage{url}
\usepackage{multicol}
\usepackage[bookmarks=true]{hyperref}
\usepackage{verbatim}
\usepackage{color}
\usepackage{tabularx}
\usepackage{booktabs}
\usepackage{listings}
\usepackage[utf8]{inputenc}
\usepackage{pifont}
\usepackage{mathtools}
\usepackage{breqn}
\usepackage{xcolor}
\usepackage{setspace}

\usepackage{atbegshi} % to hook to shipout

\newcommand{\cmark}{\ding{51}}%
\newcommand{\xmark}{\ding{55}}%

%\makeatletter
%\def\endthebibliography{%
%  \def\@noitemerr{\@latex@warning{Empty `thebibliography' environment}}%
%  \endlist
%}
%\makeatother

\input{00-settings}

% ### uncomment to produce preprint IEEE copyright notice ###
%\AtBeginShipout{%
%  \AtBeginShipoutUpperLeft{%
%    \put(\dimexpr.5\paperwidth-9cm\relax,-1cm){%
%      \parbox{\textwidth}{\centering\footnotesize%
%        © 2023 IEEE. Personal use of this material is permitted. Permission from IEEE must be obtained for all other uses, in any current or future media, including reprinting/republishing this material for advertising or promotional purposes, creating new collective works, for resale or redistribution to servers or lists, or reuse of any copyrighted component of this work in other works.}%
%    }%
%  }%
%}

\begin{document}
	
\title{\LARGE \bf
A Modular Ecosystem for Research in Learning-based Receding Horizon Control
}

\author{
  \authorblockN{%
    Andrea Patrizi\authorrefmark{1}\authorrefmark{2},
    and Nikos G. Tsagarakis\authorrefmark{1}
  }
}

\maketitle

\begingroup\renewcommand\thefootnote{\authorrefmark{2}}
	\footnotetext{Department of Informatics, Bioengineering, Robotics and Systems Engineering, Università di Genova, Via All'Opera Pia 13, 16145 Genova.}
\endgroup

\begingroup\renewcommand\thefootnote{\authorrefmark{1}}
	\footnotetext{Humanoids and Human-Centred Mechatronics (HHCM), Istituto Italiano di Tecnologia (IIT), Via San Quirico 19d, 16163 Genova.}
\endgroup

\setlength{\textfloatsep}{12.0pt plus 8.0pt minus .0pt}

\begin{abstract}
Robotics research in manipulation and locomotion is undergoing a transformative shift towards the use AI-based tools, where purely learning-based control policies and pipelines are starting to take over the field. Despite being shown to be capable of remarkable robustness and performance even when applied to real-world environments, some key inherent limitations such as safety guarantees, interpretability, etc.. persist. For this reason, it is the authors' belief that more classical control approaches should not be considered outdated yet. We thus advocate for a hybrid approach, combining offline data-based policy design, specifically through Reinforcement Learning (RL), with classical online Motion Planning, i.e. Receding Horizon Control (RHC). Even though this kind of hybrid approaches are not entirely new, to the authors' knowledge, there is no specific tool currently available for search in this domain. To this purpose, we developed a modular software ecosystem, hereby synthetically presented in its main components and features. We care to stress that the framework is currently under active development, and features might not be stable or could be lacking. To facilitate usability and diffusion, we made all the core components open source under the GPLv2 license. To showcase the potential of the framework, we furthermore briefly present a proof-of-concept example combining a high-level RL agent coupled with a lower-level MPC controller for the execution of a simple locomotion task on our wheeled quadruped robot Centauro.
\end{abstract}

\IEEEpeerreviewmaketitle

\section{Background: Trajectory Optimization, Deep Reinforcement Learning and computing advancements}
\cite{rl:bellman1957markovian}
\cite{rl:bellman1960dynamic}
\cite{rl:rumelhart1986learning}
\cite{rl:kakade2001natural}
\cite{rl:peters2005natural}
\cite{rl:degris2012off}
\cite{rl:schulman2015trust}
\cite{rl:schulman2017proximal}
\cite{rl:pardo2018time}
\cite{rl:haarnoja2018soft}

\cite{frameworks::horizon_to}


\section{Learning-based MPC and research tools}
\cite{frameworks:mittal2023orbit}
\cite{frameworks:howell2022}
\cite{mpc_learn:hewing2020learning}

\section{Framework overview and rationale}

\begin{figure}[t]
	\centering
	\includegraphics[width=1.0\columnwidth]{imgs/cocluster_arch.pdf}
	\caption{AAAAA}
	\label{fig:coclbridge_arch}
\end{figure}
\cite{mystuff::lrhccontrol}
\cite{mystuff::omnirobogym}
\cite{mystuff::coclusterbridge}
\cite{mystuff::rhcviz}
\cite{mystuff::sharsoripcpp}

\begin{figure*}[t]
	\centering
	\vspace{0.1cm}
	\includegraphics[width=0.98\textwidth]{imgs/learning_based_rhc.pdf}
	\caption{AAA}
	\label{fig:lrhc_arch}
\end{figure*}

\section{Challenges and future work}

\clearpage
\bibliographystyle{ieeetr}
\bibliography{bibliography/refs}

\end{document}
