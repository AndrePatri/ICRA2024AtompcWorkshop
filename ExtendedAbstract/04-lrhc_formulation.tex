\section{A hybrid approach: \textnormal{\textit{Learning-Based Receding Horizon Control with Reinforcement Learning}}}
Most of the currently employed control tools rely either on online ``model-based'' controllers~\cite{modern_mpc:neunert2018whole,web::atlas_grip_boston_dyn} (e.g. Receding Horizon Control) or ``model-free'' learned policies (e.g. Reinforcement Learning)~\cite{mpc_learn:aswani2012provably, mpc_learn:terzi2018learning, mpc_learn:soloperto2018learning, rl:schneider2023learning, rl:miki2024learning,mpc_learn:berkenkamp2016safe,mpc_learn:marco2016automatic,mpc_learn:brunner2015stabilizing,mpc_learn:rosolia2019learning,mpc_learn:englert2017inverse,mpc_learn:koller2018learning,mpc_learn:wabersich2021probabilistic,mpc_learn:gillulay2011guaranteed,mpc_learn:wabersich2018safe,mpc_learn:berkenkamp2017safe}, which are usually trained offline. Both approaches are historically deeply rooted in DP and MDPs. 
In the past years there have been several attempts at combining-learning based method and receding horizon controllers~\cite{mpc_learn:tsounis2020deepgait,mpc_learn:gangapurwala2021real}. Specifically, the following main approaches can be identified~\cite{mpc_learn:hewing2020learning}:
\begin{itemize}
	\item[1)] \textit{Model augmentation}: integration of learned models into RHC controllers to improve prediction accuracy and control performance~\cite{mpc_learn:aswani2012provably,mpc_learn:terzi2018learning,mpc_learn:soloperto2018learning}.
	\item[2)] \textit{Adaptive tuning} and \textit{parameter optimization}: RHC parameters tuning (e.g. weights, costs, constraints), based on real-time data~\cite{mpc_learn:berkenkamp2016safe,mpc_learn:marco2016automatic,mpc_learn:brunner2015stabilizing,mpc_learn:rosolia2019learning,mpc_learn:englert2017inverse}.
	\item[3)] \textit{Safety}: a learned-policy is coupled with a RHC controller, which in this context takes the role of a \textit{safety filter}~\cite{mpc_learn:koller2018learning,mpc_learn:wabersich2021probabilistic,mpc_learn:gillulay2011guaranteed,mpc_learn:wabersich2018safe,mpc_learn:berkenkamp2017safe}.
\end{itemize}

\begin{figure*}[t]
	\centering
	\vspace{0.1cm}
	\includegraphics[width=0.98\textwidth]{imgs/learning_based_rhc.pdf}
	\caption{Our take on Learning-based Receding Horizon Control: a MPC controller is exposed to a RL agent through key runtime parameters, like contact phases, its internal state (costs, constrains..) and interfaces for setting task commands. The agent learns to exploit the underlying RHC controller to perform the tracking of user-specified high-level task references. This allows to both tackle problems which are non-trivial at the MPC level (like phase selection), while also exploiting the flexibility of the agent to complete tasks and the capability of the MPC of ensuring safety.}
	\label{fig:lrhc_arch}
\end{figure*}